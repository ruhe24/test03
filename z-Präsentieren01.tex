\documentclass[12pt,ngerman,parskip=full]{scrreprt}

\usepackage[utf8]{inputenc}
\usepackage[T1]{fontenc}

\usepackage{babel}
\usepackage{blindtext}
\usepackage{microtype}
\usepackage{xcolor}
\usepackage{csquotes}
\usepackage{paralist}
\usepackage{graphicx}

\usepackage{hyperref}
\hypersetup{
bookmarks=true, % show bookmarks bar
unicode=false, % non - Latin characters in Acrobat’s bookmarks
pdftoolbar=true, % show Acrobat’s toolbar
pdfmenubar=true, % show Acrobat’s menu
pdffitwindow=false, % window fit to page when opened
pdfstartview={FitH}, % fits the width of the page to the window
pdftitle={My title}, % title
pdfauthor={Author}, % author
pdfsubject={Subject}, % subject of the document
pdfcreator={Creator}, % creator of the document
pdfproducer={Producer}, % producer of the document
pdfkeywords={keyword1, key2, key3}, % list of keywords
pdfnewwindow=true, % links in new window
colorlinks=true, % false: boxed links; true: colored links
linkcolor=blue, % color of internal links
filecolor=blue, % color of file links
citecolor=blue, % color of file links
urlcolor=blue % color of external links
}

\newcommand{\person}[1]{\textsc{\textcolor{red}{#1}}}

\author{von Ruth}
\title{Übungsfile Geburtstag\\
\person{Für einen sehr besonderen Menschen:} \\
Liebe/Lieber ...\\
per \LaTeX}
\date{Ort, Dezember 2021}

\begin{document}
\maketitle

\tableofcontents

\listoffigures

\begin{center}
\includegraphics[width=.85\textwidth]{Bilder/Blumbutterfly1}\captionof{figure}{Gelber Schmetterling mit Blumen}\label{fig:Blumbutterfly1}
\end{center}

\chapter{Besonderer Tag}
\person{\textbf{Liebe ...}}, heute ist ein besonderer Tag. 
\section{Dein Geburtstag}
Der Tag deiner Geburt jährt sich. 
\subsection{Quellen}
So kam es mir jedenfalls zu Ohren. 
\subsection{Alter}
Danach sollst du heute ... Jahre alt werden. 
\section{In Wirklichkeit}
Aber die Jahre sind das Eine, das Wesentliche ist unsichtbar, zeitlos und ewig jung. Also gibt es in Wirklichkeit keinen Geburtstag oder gibt es immer Geburtstag. 

\chapter{Viel erlebt und erfahren}
Du hast in deinem Leben vieles und sehr Unterschiedliches erlebt und erfahren. \\
\section{Schönes}
...
\section{Nicht so Schönes}
Was dich so besonders macht ist, ... 
\subsection{Unzählige Situationen}
...
\subsection{Motto}
\enquote{Bleib deinem Herzen treu.} ...
\section{Was sich parallel durch dein Leben zog .... }
... 
\section{So kennen es auch Familie und Freunde}
\marginpar{Familie und Freunde waren dir immer wichtig} 

\begin{center}
\includegraphics[width=0.9\textwidth]{Bilder/Schmetterling2}\captionof{figure}{Ein Schmetterling auf roter Blumenpracht}\label{fig:Schmetterling2}
\end{center}

\begin{center}
\includegraphics[width=0.7\textwidth]{Bilder/Babyhand2}\captionof{figure}{Vertrauen wie ein Kind}\label{fig:Babyhand1}
\end{center}

\marginpar{Balance zwischen Aktivität und Vertrauen} 

\chapter{Dank}
Der Dank gehört \textbf{dir, liebe/lieber ...}. 
\section {Danke für so vieles}
... 
\section{Aufzählung einer kleiner Auswahl}
Von den zahlreichen Situationen usw. seien hier insbesondere erwähnt:  
\begin{enumerate}
\item Gespräche, 
\item wunderschöne Wanderungen in den Bergen, 
\item gemeinsames Lachen und  
\item Ehrlichkeit und Offenheit
\end{enumerate}
\section{Widmung}
In Dankbarkeit ist dieses \LaTeX-Domument dir gewidmet.
\section{Wünsche}
Zum Geburtstag und für alles Weitere meine herzlichsten Wünsche.

\begin{center}
\includegraphics[width=0.5\textwidth]{Bilder/Schneeweihmann01}\captionof{figure}{Ein glücklicher Weihnachtsschneemann}\label{fig:Schneeweihmann01}
\end{center}

\begin{center}
\includegraphics[width=0.4\textwidth]{Bilder/Fuchs1}\captionof{figure}{Immer wachsam unser Gartenchef Schlaufuchs}\label{fig:Fuchs1}
\end{center}

\end{document}